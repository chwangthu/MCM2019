%%%%%%%%%%%%%%%%%%%%%%%%%%%%%%%%%%%%%%%%%
%%            请在此填写摘要            %%
%% 请勿编译/排版此文件,请编译PAPER.tex!  %%
%%%%%%%%%%%%%%%%%%%%%%%%%%%%%%%%%%%%%%%%%
\begin{abstract}\small
In order to identify the locations where specific opioid use started and describe the patterns and characteristics of opioid and heroin incidents spread, our team analyze, process and construct models on the drug reports and census profiles. Then, according to the applicable range of the model, we make suggestions to the government in the short and long term respectively.
\vspace{5pt}

Specifically, in the data process, we use the K-means algorithm to combine geographically close counties into one cluster. The benefit of clustering is to reduce the influence of the volatility of certain county’s drug reports, while improving the accuracy of location identification. For census data, every variable can be classified into certain socio-economic factor by its name (e.g. ANCERSTRY including estimate, estimate margin of error, percent, etc.) Since we have no idea whether the estimate number or the percent number provide better information for our modeling, it's reasonably to use the Entropy Weight Method to combine variables into factors for further use.
\vspace{5pt}

For Part 1, we build the Logistic Model to identify the start time of specific opioid use in each cluster. We conclude that the Cluster-8 in Pennsylvania first started using heroin drugs and Cluster-3 in Kentucky first use synthetic opioids. However, the Logistic Model doesn't take the spread of heroin incidents into account. Therefore, we add the interaction of different clusters into the LM to get the Lotka-Volterra Model and forecast the next 5 years trends. We use the upper bound calculated from LM as the drug identification threshold levels, and find Cluster-2 most in West Virginia and Cluster-7 in Pennsylvania will breakthrough threshold in 2018 and 2021. It's urgent for the government to adopt effective measures.
\vspace{5pt}

For Part 2, after combining variables into factors, we find certain factors have a strong correlation with the drug reports in different clusters. In order to select the best influence factors for each cluster, a correlation matrix is calculated and contributes to the growth in opioid use and addiction are partly explained. We build linear regression model to regress the natural change rate on factors and modify the change rate in the Lotka-Volterra by taking the weighted average of two results. For the simple Lotka-Volterra Model having chaos phenomenon, the Modified Lotka-Volterra is more stable to forecast the next 50 years trends.
\vspace{5pt}

For Part 3, combining the insights from the model, we identify the short-term and long-term policies respectively. For the short term, our main goal is to reduce the natural change rate of each clusters in the Lotka-Volterra. Hence, we suggest the government in each state adopt tough policing policies to crack down on the illicit sale of heroin. For the long term, our main goal is to change the socio-factors in the linear regression model. We suggest the government to increase the regional education attainment and decrease the unmarried rate, which will ultimately reduce the opioid users in the long run.
\vspace{5pt}
% 美赛论文中无需注明关键字。若您一定要使用,
% 请将以下两行的注释号'%'去除,以使其生效;
% 若您不使用,可直接将这段注释删除
% \vspace{5pt}
% \textbf{Keywords}: MATLAB, mathematics, LaTeX.

\end{abstract}




%%%%%%%%%%%%%%%%%%%%%%%%%%%%%%%%%%%%%%%%%%
% 如不理解以下部分中各命令的含义,请勿修改! %
%%%%%%%%%%%%%%%%%%%%%%%%%%%%%%%%%%%%%%%%%%

%---------以下生成sheet页----------
% 下面的语句可调整全文行距为标准值的0.6倍,请自行使用
% \renewcommand{\baselinestretch}{0.6}\normalsize
\maketitle  			% 生成sheet页
\thispagestyle{empty}   % 不要页眉页脚和页码
\setcounter{page}{-100} % 此命令仅是为了避免页码重复报错,不要在意

%---------以下生成目录----------
\newpage
\tableofcontents
\thispagestyle{empty}   % 不要页眉页脚和页码
\newpage

%---------以下生成正文----------
\setlength\parskip{0.8\baselineskip}  % 调整段间距
\setcounter{page}{1}    % 从正文开始计页码
\pagestyle{fancy}