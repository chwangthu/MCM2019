%%%%%%%%%%%%%%%%%%%%%%%%%%%%%%
% 	   美赛模板,正文部分		 
%          PAPER.tex         
%%%%%%%%%%%%%%%%%%%%%%%%%%%%%%

\documentclass[12pt]{article}

% 请在此填写控制号、题号和标题,年份不需要填(自动以当前电脑时间年份为准)
\usepackage[1923231]{easymcm}\problem{C}   
\usepackage{palatino} % 这个是COMAP官方杂志采用的字体,如不需要可注释掉,以使用默认字体

\title{\large God or evil? A ... model analysing and predicting opioid use}  % 标题
\setlength\parindent{0pt} %default no indent
\newcommand{\upcite}[1]{\textsuperscript{\textsuperscript{\cite{#1}}}}
% 如您参加的是ICM(即选择了D/E/F题),请使用以下的命令修改Summary Sheet题头
% \renewcommand{\contest}{Interdisciplinary Contest in Modeling (ICM) Summary Sheet}

% 正文开始
\begin{document}
%%%%%%%%%%%%%%%%%%%%%%%%%%%%%%%%%%%%%%%%%
%%            请在此填写摘要            %%
%% 请勿编译/排版此文件,请编译PAPER.tex!  %%
%%%%%%%%%%%%%%%%%%%%%%%%%%%%%%%%%%%%%%%%%
\begin{abstract}\small
In order to identify the locations where specific opioid use started and describe the patterns and characteristics of opioid and heroin incidents spread, our team analyze, process and construct models on the drug reports and census profiles. Then, according to the applicable range of the model, we make suggestions to the government in the short and long term respectively.
\vspace{5pt}

Specifically, in the data process, we use the K-means algorithm to combine geographically close counties into one cluster. The benefit of clustering is to reduce the influence of the volatility of certain county’s drug reports, while improving the accuracy of location identification. For census data, every variable can be classified into certain socio-economic factor by its name (e.g. ANCERSTRY including estimate, estimate margin of error, percent, etc.) Since we have no idea whether the estimate number or the percent number provide better information for our modeling, it's reasonably to use the Entropy Weight Method to combine variables into factors for further use.
\vspace{5pt}

For Part 1, we build the Logistic Model to identify the start time of specific opioid use in each cluster. We conclude that the Cluster-8 in Pennsylvania first started using heroin drugs and Cluster-3 in Kentucky first use synthetic opioids. However, the Logistic Model doesn't take the spread of heroin incidents into account. Therefore, we add the interaction of different clusters into the LM to get the Lotka-Volterra Model and forecast the next 5 years trends. We use the upper bound calculated from LM as the drug identification threshold levels, and find Cluster-2 most in West Virginia and Cluster-7 in Pennsylvania will breakthrough threshold in 2018 and 2021. It's urgent for the government to adopt effective measures.
\vspace{5pt}

For Part 2, after combining variables into factors, we find certain factors have a strong correlation with the drug reports in different clusters. In order to select the best influence factors for each cluster, a correlation matrix is calculated and contributes to the growth in opioid use and addiction are partly explained. We build linear regression model to regress the natural change rate on factors and modify the change rate in the Lotka-Volterra by taking the weighted average of two results. For the simple Lotka-Volterra Model having chaos phenomenon, the Modified Lotka-Volterra is more stable to forecast the next 50 years trends.
\vspace{5pt}

For Part 3, combining the insights from the model, we identify the short-term and long-term policies respectively. For the short term, our main goal is to reduce the natural change rate of each clusters in the Lotka-Volterra. Hence, we suggest the government in each state adopt tough policing policies to crack down on the illicit sale of heroin. For the long term, our main goal is to change the socio-factors in the linear regression model. We suggest the government to increase the regional education attainment and decrease the unmarried rate, which will ultimately reduce the opioid users in the long run.
\vspace{5pt}
% 美赛论文中无需注明关键字。若您一定要使用,
% 请将以下两行的注释号'%'去除,以使其生效;
% 若您不使用,可直接将这段注释删除
% \vspace{5pt}
% \textbf{Keywords}: MATLAB, mathematics, LaTeX.

\end{abstract}




%%%%%%%%%%%%%%%%%%%%%%%%%%%%%%%%%%%%%%%%%%
% 如不理解以下部分中各命令的含义,请勿修改! %
%%%%%%%%%%%%%%%%%%%%%%%%%%%%%%%%%%%%%%%%%%

%---------以下生成sheet页----------
% 下面的语句可调整全文行距为标准值的0.6倍,请自行使用
% \renewcommand{\baselinestretch}{0.6}\normalsize
\maketitle  			% 生成sheet页
\thispagestyle{empty}   % 不要页眉页脚和页码
\setcounter{page}{-100} % 此命令仅是为了避免页码重复报错,不要在意

%---------以下生成目录----------
\newpage
\tableofcontents
\thispagestyle{empty}   % 不要页眉页脚和页码
\newpage

%---------以下生成正文----------
\setlength\parskip{0.8\baselineskip}  % 调整段间距
\setcounter{page}{1}    % 从正文开始计页码
\pagestyle{fancy}		% 摘要请到ABSTRACT.tex中填写

\section*{\centering Memo}
\newpage

\section{Introduction}
\subsection{Problem Background}
Opioids, no matter synthetic ones or non-synthetic ones, play an significant role in our daily life. The proper utilization of opioids helps thousands of people get relieved from their pains. However, misuse of them can really destroy the society. Therefore, it is essential and urgent to gain a better understanding of opioids' spread and influencing factors.

Recent years have witnessed the tremendous growth of opioids cases. How this trend will develop remains an national concern. With data of annual drug use and socio-economic factors in five U.S. states over the past few years, we perform data mining and construct a model to figure out the pattern of opioids use and what contributes to it.

% \subsection{Literature Review}
% A literatrue\upcite{1} say something about this problem ...

\subsection{Our work}
We do such things ...

\begin{enumerate}[\bfseries 1.]
    \item We do ...
    \item We do ...
    \item We do ...
\end{enumerate}

\section{Assumptions and Nomenclature}
\subsection{Assumptions}
Given the lack of data and limitation of our knowledge, we made the following assumption:
\begin{itemize}
    \item
\end{itemize}

\subsection{Notations}
The primary notations used in this paper are listed in \textbf{Table \ref{tb:notation}}.
\begin{table}[!htbp]
\begin{center}
\caption{Notations}
\begin{tabular}{cl}
	\toprule
	\multicolumn{1}{m{3cm}}{\centering Symbol}
	&\multicolumn{1}{m{8cm}}{\centering Definition}\\
	\midrule
	$A$&the first one\\
	$b$&the second one\\
	$\alpha$ &the last one\\
	\bottomrule
\end{tabular}\label{tb:notation}
\end{center}
\end{table}

\section{Data Processing}
We are provided with two types of data, the number of drug reports in categories and socio-economic factors. Both of them are specific to counties. The original data contains a lot of redundant and invalid items, which can seriously affect the accuracy and versatility of our model. Thereby, we first apply some data processing techniques before we construct the model.
\subsection{Missing Data Processing}
% In US, each county is assigned a unique five-digit code called FIPS county code. 
In the worksheet about socio-economic factors, there are many cells that contain no valid information at all. We delete columns and rows which contains a lot of useless cells. For other columns and rows, when invalid items appear, we replace it with the average value of the entire sequence.

\subsection{Clustering}
Since we have detailed number of drug reports, it is not hard to fit a curve to observe the trend of opioid use for each state and for each county. However, we consider this method inappropriate. On the one hand, the sample for only one county is normally fluctuating irregularly, i.e. drug reports of heroin in ACCOMACK, VA was 2, 38, 6 during 2012-2014. It is unnecessary and meaningless to bulid a model when facing such fluctuation. On the other hand, the trend of drug reports in the five states is not obvious. This can be seen in the figure below. %TODO: Add figure

Considering that state and county are not suitable for our research, we decide to introduce a unit in between. We put counties on a two-dimensional plane and regard them as nodes. Then, k-means clustering method\upcite{1}, which aims to partion nodes into k clusters minimizing the within-cluster sum of squares, was performed. In our case, we take $k = 10$. The distance of two counties can be computed with their latitude and longtitude retrieved from U.S. Census Bureau\upcite{2}. %TODO: Add cluster figure

\subsection{}

\section{The Models}
\subsection{Model 1}
\subsubsection{Detail 1 about Model 1}
\begin{equation}
    e^{i\theta}=\cos\theta+i\sin\theta.
\end{equation}

\section{Strengths and Weaknesses}
\subsection{Strengths}
\begin{itemize}
    \item Our model takes the idea of clustering, which overcomes the difficulty of analysing either the whole state or one single county.
    \item We take several measures to process the dataset and make it convenient for use, while the authenticity and integrity of data is still maintained.
\end{itemize}

\subsection{Weaknesses}
\begin{itemize}
    \item Nothing yet
 \end{itemize}

\begin{thebibliography}{99}
\addcontentsline{toc}{section}{References}  %引用部分标题("Refenrence")的重命名
\bibitem{2}Gazetteer Files - Geography - U.S. Census Bureau. \texttt{\\https://www.census.gov/geo/maps-data/data/gazetteer.html}
% \bibitem{3}Elisa T. Lee, Oscar T. Survival Analysis in Public Health Research. \emph{Go. College of Public Health}, 1997(18):105-134.
\bibitem{1}MacQueen, J. (1967, June). Some methods for classification and analysis of multivariate observations. In Proceedings of the fifth Berkeley symposium on mathematical statistics and probability (Vol. 1, No. 14, pp. 281-297).
\bibitem{3}DrugBank. \texttt{https://www.drugbank.ca/drug}
\end{thebibliography}


% ==============以下为附录内容,如您的论文中不需要程序附录请自行删除====================
\clearpage
\begin{subappendices}						% 附录环境
\section*{Apendix: The Source Codes}		% 附录标题可以自行修改
\addcontentsline{toc}{section}{Appendix}  	% 将附录内容加入到目录中

This python program implements k-means clustering.
\lstinputlisting[language={python}, caption=\texttt{k-means.py}]{../code/k-means.py}
% \inputpython{../code/k-means.py}{0}{68}

This MATLAB program is used to calculate the value of variable $a$.
\begin{lstlisting}[language={Matlab}, caption=\texttt{temp.m}]
a = 0;
for i = 1:5
	a = a + 1;
end
\end{lstlisting}

This LINGO program is used to search the optimize solution of 0-1 problem.
\begin{lstlisting}[language=Lingo, caption=\texttt{temp.lg4}]
model:
sets:
WP/1..12/: M, W, X;
endsets
data:
M = 2 5 18 3 2 5 10 4 11 7 14 6;
W = 5 10 13 4 3 11 13 10 8 16 7 4;
enddata
max = @sum(WP:W*X);
@sum(WP: M * X)<=46;
@for(WP: @bin(X));
end
\end{lstlisting}

\textbf{\textcolor[rgb]{0.98,0.00,0.00}{Input matlab source:}}
\lstinputlisting[language=Matlab]{../code/matlab-test.m}

\end{subappendices}
% =================================================================================



\end{document}